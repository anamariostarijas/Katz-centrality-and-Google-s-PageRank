% PREAMBULA DOKUMENTA
\documentclass[a4paper]{article}
\usepackage[slovene]{babel}
\usepackage[utf8]{inputenc}
\usepackage[T1]{fontenc}
\usepackage{graphicx}
\usepackage {lmodern}
\usepackage{amsfonts}
\usepackage{amsmath}
\usepackage{amssymb}
\usepackage{tikz}
\usepackage{mathtools}
\usepackage{wrapfig}
\usepackage{commath}



\begin{document}
\thispagestyle{empty}

\begin{figure}[t]
\begin{center} 
\includegraphics[width=6cm]{fmf.png}\\[4cm]
\end{center}
\end{figure}

\begin{center}
\Huge\textbf{Katzova središčnost in Googlov PageRank}\\[0.5cm]
\large\textsc{Kratko poročilo, Projekt pri predmetu Finančni praktikum}\\[4cm]
\end{center}

\begin{flushleft}

\end{flushleft}
\vspace{\fill}
\begin{flushright}
Anamari Oštarijaš, Tina Ražić
\end{flushright}

\begin{flushleft}
Ljubljana, december 2018
\end{flushleft}

\newpage
\tableofcontents

\newpage
\section{Opis projekta}
\hspace{4.8mm}Kompleksna omrežja lahko analiziramo z uporabo različnih kvantitetnih merjenj, imenujemo jih tudi mere središčnosti,  ki intuitivno zajamejo pomembnost določenih vozlišč. 
V projektu bova implementirali Googlov PageRank in Katzovo središčnost z uporabo potenčne metode. Na različnih grafih (tudi socialnih omrežjih) bova analizirali in primerjali, kako merjenji razvrstita vozlišča po pomembnosti. 

\section{Katzova središčnost}

<<<<<<< HEAD:Teorija/Kratko_porocilo_pagerank.tex
\hspace{4.8mm}Katzova središčnost izmeri vpliv igralca v omrežju tako, da upošteva direktne sosede igralca in vse druge igralce, ki so posredno povezani s tem igralcem preko njegovih direktnih sosedov. \\
Naj bo naše omrežje graf z $n$ vozlišči oziroma igralci. Vsaka povezava v grafu dobi utež $\alpha$ in z $\alpha^{d}$ izračunamo težo povezave vozlišča z drugim vozliščem, pri čemer je $d$ število povezav med njima. Naš graf predstavimo z matriko sosednosti A, torej element matrike $a_{ij}$ ima vrednost $1$, če je vozlišče $i$ povezano z vozliščem $j$, in $0$, če nista povezana. Potence matrike A nam povejo, če je vozlišče povezano s drugimi indirektnimi vozlišči preko sosedov. Na primer, če je v matriki $A^{3}$ element $a_{2,5}  = 1,$ pomeni, da sta vozlišče $2$ in vozlišče $5$ povezana s tremi povezavami preko sosedov prve stopnje in sosedov druge stopnje. \\
=======
Katzova središčnost izmeri vpliv igralca v omrežju, tako da upošteva direktne sosede igralca in vse druge igralce, ki so posredno povezani s tem igralcem, preko njegovih direktnih sosedov. Naj bo naše omrežje graf z $n$ vozlišči oziroma igralci. Vsaka povezava v grafu dobi utež $\alpha$ in z $\alpha^{d}$ izračunamo težo povezave vozlišča z drugim vozliščem, pri čemer je $d$ število povezav, ki ju povezuje. Naš graf predstavimo z matriko sosednosti A, torej element matrike $a_{ij}$ ima vrednost $1$, če je vozlišče $i$ povezano z vozliščem $j$ in $0$, če nista povezana. Potence matrike A nam povejo, če je vozlišče povezano s drugimi indirektnimi vozlišči preko sosedov. Na primer: če je v matriki $A^{3}$ element $a_{2,5}  = 1,$pomeni, da sta vozlišče $2$ in vozlišče $5$ povezana s tremi povezavami preko sosedov prve stopnje in sosedov druge stopnje. \\
>>>>>>> 48d367efa96b619833d4699dcc0689241241161b:Kratko_porocilo_pagerank.tex
Označimo s $C_{Katz}(i)$ Katzovo središčnost vozlišča $i$. Potem lahko izračunamo središčnost na sledeči način:

$$C_{Katz}(i) = \sum_{k=1}^{\infty}\sum_{j=1}^{n}\alpha^{k}(A^{k})_{ij}.$$

Pri izbiri $\alpha$ moramo upoštevati zgornjo omejitev $$\alpha < \frac{1}{\abs{\lambda_{max}}}.$$


\section{Googlov PageRank}
\hspace{4.8mm}Ta metoda temelji na predpostavki, da število linkov (povezav)  do in iz strani daje informacijo o pomembnosti strani. \\
Naj bodo vse spletne strani urejene s števili od $1$ do $n$ in naj bo $i$ neka spletna stran. Potem $O_i$ določa množico strani, s katerimi je i povezana, tako da $i$ vsebuje link do strani v množici $O_i$ (\textit{outlink}). Število outlinkov označimo z $N_i = \|O_i\|$. Množica \textit{inlinkov}, označena z $I_i$, je množica strani, ki imajo outlink do $i$ (strani v $I_i$ vsebujejo linke do $i$).
Splošno, več ko ima stran $i$ inlinkov, pomembnejša je. \\  Da bi preprečili možne manipulacije, definiramo rang vozlišča $i$ tako, da če ima visoko rangirana stran $j$ outlink do $i$, to doda pomembnosti $i$ na sledeč način: rang strani $i$ je utežena vsota rangov strani, ki imajo outlink do $i$. Obteženost  je taka, da je rang strani $j$ razdeljen enakomerno med njenimi outlinki. Z enačbo: $$r_i = \sum_{j \in I_1} \frac{r_j}{N_j}.$$ \\
Ta definicija je rekurzivna, zato pageranki ne morejo biti izračunani direktno. Uporabimo iteracijo. Najprej ugibamo začetni rangni vektor $r^0$. Potem iteriramo:
$$r_i^{(k+1)} = \sum_{j \in I_1} \frac{r_j^{(k)}}{N_j}$$ \\
Raje zapišimo problem z matrikami. Naj bo $Q$ kvadratna matrika dimenzije $n$. Definiramo:
\[
Q_{ij} = 
\left \{
	\begin{array}{ll}
		1/N_j  &, \mbox{če obstaja link od j do i }  \\
		0 &, \mbox{sicer} 
	\end{array}
\right. \]
\\
Torej ima vrstica $i$ neničelne elemente na mestih inlinkov $i$. Podobno ima stolpec $j$ neničelne elemente enake $N_j$ na mestih  outlinkov $j$, vsota vseh elementov v stolpcu je enaka $1$. \\
Enačbo sedaj lahko zapišemo v matrični obliki:
$$ \lambda r = Qr,     \qquad \lambda = 1,$$
Tako dobimo, da je $r$ lastni vektor matrike $Q$ z lastno vrednostjo $\lambda = 1$. Sedaj preprosto vidimo, da je iteracija ekvivalentna
$$r^{(k+1)} = Qr^{(k)},\qquad  k=0,1,… ,$$
kar je potenčna metoda za izračun lastnega vektorja.\\
\\ Predpostavili smo, da obstaja lastna vrednost enaka 1. Ta predpostavka je pojasnjena s teorijo slučajnih sprehodov, podrobnosti so napisane v daljši dokumentaciji (\textit{katz\_google\_pagerank\_definition.pdf}). \\

\subsection{Prilagoditev modela}\hspace{4.8mm} Da naša matrika zadošča pogojem \textit{Perron-Frobeniusovega izreka}, torej da je le ena lastna vrednost enaka 1, jo moramo ustrezno prilagoditi. Matrika mora biti ireducibilna, pozitivna in desna stohastična.\\
 Ničelne stolpce v matriki $Q$ nadomestimo s konstantnimi vrednostmi na vseh mestih (s tem rešimo problem strani, ki ne vsebujejo nobenih outlinkov). 
\\Definiramo vektorje: \\
\[
d_j = 
\left \{
	\begin{array}{ll}
		1  &, \mbox{če}\hspace{0.8mm} N_j = 0 \\
		0 &, \mbox{sicer} 
	\end{array}
\right. \]
za $j = 1, .., n$ in
$$e = [1 … 1] ^T \in \mathbb{R}^n.$$
<<<<<<< HEAD:Teorija/Kratko_porocilo_pagerank.tex
Prilagojena matrika je definirana s $ P = Q + \frac{1}{n}ed^T$, to je desna stohastična matrika, ki zadošča:
$$e^TP = e^T$$.
Želimo definirati pagerank vektor kot enoličen lastni vektor matrike $P$ z lastno vrednostjo $1$:
$$Pr=r$$. \\
Element $r_i$ je verjetnost, da po velikem številu korakov slučajni uporabnik omrežja pristane na strani $i$.\\
=======
Prilagojena matrika je definirana s $ P = Q + \frac{1}{n}ed^T$,desno stohastično matrika, ki zadošča:
$$e^TP = e^T.$$
Želimo definirati pagerank vektor kot enoličen lastni vektor matrike $P$ z lastno vrednostjo $1$:
$$Pr=r.$$\\
Element $r_i$ je verjetnost, da po velikem številu korakov slučajni uporabnik pristane na strani $i$.\\
>>>>>>> 48d367efa96b619833d4699dcc0689241241161b:Kratko_porocilo_pagerank.tex
Zaradi velikosti spleta je matrika linkov $P$ reducibilna, torej pagerank lastni vektor ni dobro definiran. Da si zagotovimo ireducibilnost, umetno dodamo linke iz vsake spletne strani do vseh drugih. To lahko storimo, če vzamemo konveksno kombinacijo $P$ in matrike ranga 1:
$$A=\alpha P + (1-\alpha)\frac{1}{n}ee^T,$$
za nek $\alpha$, ki zadošča $0 \leq \alpha \leq 1$. Matrika $A$ je desna stohastična. Razlaga naključnega sprehoda dodatnega rang-1 izraza je, da bo uporabnik na vsakem časovnem koraku skočil na naključno stran z verjetnostjo $1- \alpha$. Če so bile lastne vrednosti desne stohastične matrike P enake $1, \lambda_2, \lambda_3, ... , \lambda_n$, bodo lastne vrednosti matrike $A$ enake $1, \alpha \lambda_2, \alpha \lambda_3, ... , \alpha \lambda_n$. Tako si zagotovimo, da je le ena lastna vrednost enaka 1 in matrika $A$ zadošča pogojem \textit{Perron-Frobeniusovega izreka}. Matriko $A$ poznamo pod imenom \textit{Googlova matrika}.


\subsection{Potenčna metoda}
\hspace{4.8mm}Želimo rešiti problem lastne vrednosti:
$$Ar = r,$$
kjer je $r$ nomaliziran, $\|r\|_1=1$. Iskani lastni vektor označimo s $t_1$. \\
Prepostavimo, da imamo dan začetni približek $r{(0)}$. \\
\\
\textbf{Algoritem:} \\
\begin{center} za k = 1, 2, … do konvergence
	$$q^{(k)} = Ar^{(k-1)}$$ 
	$$r^{(k)} = q^{(k)}/ \|q^{(k)}\|_1$$ \\
\end{center}
Konvergenca je odvisna od porazdelitve lastnih vrednosti. Če je druga največja lastna vrednost blizu 1, bo iteracija zelo počasna. To na srečo ne velja za Google matriko. Vektor normaliziramo, da ne bi z iteracijami postali preveliki ali premajhni, posledično nereprezentativni s števili s plavajočo vejico. To v resnici ni potrebno, saj se v primeru desnih stohastičnih matrik temu izognemo. 
En izračun pageranka lahko traja več dni.

\section{Načrt dela}
\hspace{4.8mm}Najprej bova oba algoritma preizkusili na manjših grafih za zagotovitev pravilnosti, kasneje pa še na omrežjih. Primerjali bova razlike v rangiranju in čase izračunov na večjih in manjših grafih, da vidiva, kako velikost omrežja vpliva na vsako metodo. \\
Algoritma bova napisali v programu Python s pomočjo knjižnice NetworkX za generiranje omrežij. 




\end{document}