% PREAMBULA DOKUMENTA
\documentclass[a4paper]{article}
\usepackage[slovene]{babel}
\usepackage[utf8]{inputenc}
\usepackage[T1]{fontenc}
\usepackage{graphicx}
\usepackage {lmodern}
\usepackage{amsfonts}
\usepackage{amsmath}
\usepackage{amssymb}
\usepackage{tikz}
\usepackage{mathtools}
\usepackage{wrapfig}
\usepackage{commath}



\begin{document}
\thispagestyle{empty}

\begin{figure}[t]
\begin{center} 
\includegraphics[width=6cm]{fmf.png}\\[4cm]
\end{center}
\end{figure}

\begin{center}
\Huge\textbf{Katzova središčnost in Googlov PageRank}\\[0.5cm]
\large\textsc{Kratko poročilo, Projekt pri predmetu Finančni praktikum}\\[4cm]
\end{center}

\begin{flushleft}

\end{flushleft}
\vspace{\fill}
\begin{flushright}
Anamari Oštarijaš, Tina Ražić
\end{flushright}

\begin{flushleft}
Ljubljana, december 2018
\end{flushleft}

\newpage
\tableofcontents

\newpage
\section{Opis projekta}
\hspace{4.8mm}Kompleksna omrežja lahko analiziramo z uporabo različnih kvantitetnih merjenj, imenujemo jih tudi mere središčnosti,  ki intuitivno zajamejo pomembnost določenih vozlišč. 
V projektu bova implementirali Googlov PageRank in Katzovo središčnost z uporabo potenčne metode. Na različnih grafih (tudi socialnih omrežjih) bova analizirali in primerjali, kako merjenji razvrstita vozlišča po pomembnosti. 

\section{Katzova središčnost}
\subsection{Matematično ozadje}


\hspace{4.8mm}Katzova središčnost izmeri vpliv igralca v omrežju tako, da upošteva direktne sosede igralca in vse druge igralce, ki so posredno povezani s tem igralcem preko njegovih direktnih sosedov. \\
Naj bo naše omrežje graf z $n$ vozlišči oziroma igralci. Vsaka povezava v grafu dobi utež $\alpha$ in z $\alpha^{d}$ izračunamo težo povezave vozlišča z drugim vozliščem, pri čemer je $d$ število povezav med njima. Naš graf predstavimo z matriko sosednosti A, torej element matrike $a_{ij}$ ima vrednost $1$, če je vozlišče $i$ povezano z vozliščem $j$, in $0$, če nista povezana. Potence matrike A nam povejo, če je vozlišče povezano s drugimi indirektnimi vozlišči preko sosedov. Na primer, če je v matriki $A^{3}$ element $a_{2,5}  = 1,$ pomeni, da sta vozlišče $2$ in vozlišče $5$ povezana s tremi povezavami preko sosedov prve stopnje in sosedov druge stopnje. \\
Označimo s $C_{Katz}(i)$ Katzovo središčnost vozlišča $i$. Potem lahko izračunamo središčnost na sledeči način:

$$C_{Katz}(i) = \sum_{k=1}^{\infty}\sum_{j=1}^{n}\alpha^{k}(A^{k})_{ij}.$$

Pri izbiri $\alpha$ moramo upoštevati zgornjo omejitev $$\alpha < \frac{1}{\abs{\lambda_{max}}}.$$
\subsection{Psevdokoda}

\end{document}
